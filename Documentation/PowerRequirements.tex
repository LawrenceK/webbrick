\section{Power Requirements}
\index{Power Requirements}

 This section deals with the power budget required to run a WebBrick.
 
 \subsection{Power Supply}
 \index{Power Supply}
 
    WebBricks should be run from a power supply that delivers 12.6V to 18V.  12.6V is a minimum to achieve a full 0-10V range on
    the analogue outputs.
    
    A quiescent WebBrick will consume 55-60mA at 12.6V, however the WebBrick in real use will be supplying power to a range of external items
    including:
 	
\begin{itemize}

\item{\bf LEDs} The WebBrick may drive up to 8 mimic LEDs, each consuming 5mA. {\it 40mA}

\item{\bf Relays} There are two relays, each using 30mA to hold the contacts closed. {\it 60mA}

\item{\bf Temperature Sensors} There may be up to 5 sensors, these are driven periodically. {\it 5mA peak}

\item{\bf Analogue Outputs} There are four buffered 0-10V outputs each capable of supplying 20mA. {\it 80mA}

\item{\bf General Output Drive} The digital outputs can each supply 5mA, although only four are presented as TTL.
if a rotary encoder is connected it will use 2mA.  Driving the triac and open collector gates requires 2mA each. {\it 40mA}

\end{itemize}

Total power budget, all outputs driven, all sensors connected 280mA.
 	
 	
\subsubsection{Back Feed}
\index{Back Feed}
 	
    Because the WebBrick consumes so little current, it is possible to inadvertently 
    power it through the coils of external relays connected to 
    the open collector outputs.  This only becomes an issue if one needs to fully power 
    down a WebBrick. To avoid this situation we suggest
    that the +ve side of relay coils are driven from the same power supply as drives the WebBricks.