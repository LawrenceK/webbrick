\section{Architecture Description}
\index{Architecture Description}

\subsection{Hardware}
\index{Hardware Architecture}

The WebBrick V6 is based around the PIC 18F6x22 chip and a NetMedia SitePlayer \index{SitePlayer} to create 
both HTTP and UDP command channels. The base 8 digital outputs are buffered through a Open collector 
drivers that can handle 12V 500mA drive with reasonable protection.

Some provision is made for input protection from excessive voltage by a combination of a current limiting
resistor and the clamp on the PIC chip inputs, edge triggered inputs are de-bounced in software.

WebBrick uses several properties of the SitePlayer:

\begin{itemize}

\item

SitePlayer can send commands to the PIC ship from its own web interface.

\item

SitePlayer can send commands to the PIC chip received over HTTP and UDP.

\item

The WebBrick can get the SitePlayer to send a UDP packet, the contents of which are 
managed by the PIC chip.

\end{itemize}

\subsection{Software Architecture}
\index{Software Architecture}

The WebBrick is based around a configurable state machine, this is driven by a variety
of triggers. The triggers can be generated by digital inputs, analogue inputs, temperature sensors
and network command. These state machines are configured on what to do when they receive a trigger, this
can be as simple as switch an output on or may involve changing the current scene and sending a network
message to the home gateway.
Although there 
are direct output control commands, in general operation the user configures the inputs to perform 
{\em actions} on outputs.

{\em actions} can be activated by:

\begin{itemize}

\item

Physical input pins, for example taking digital input 0 low momentarily would set the actions configured for 
state engine 0 in motion.  There is a de-bounce time on these circuits.
\item

HTTP request. \index{HTTP GET}

\item

Analogue or temperature inputs above or below configured thresholds.

\item

UDP packet\index{UDP}
\item

Scheduled Event\index{Scheduled Event}

\end{itemize}

\subsection{WebBrick Indicator LEDs}
\index{Indicator LEDs}

	The WebBrick has three indicators, which can been seen through the top translucent window.  They are:
	
	\begin{itemize}
		\item{Red} This indicator shows that the power to the process control and web server chips is healthy.
		\item{Green} This indicator shows that the web server chip is seeing a network link
		\item{Blue} This is the WebBrick heartbeat, it blinks a lot, it has different blinks at start-up, but once
				a WebBrick is running it will settle to a 1 sec blink rate. \(see section below\)
	\end{itemize}
	

\subsection{WebBrick Start-Up}
\index{Start-up sequence}

When power is applied to the WebBrick, the SitePlayer chip and the PIC chip will start.

It is important that any WebBrick has deterministic behaviour, therefore there is a settling period
before the configuration details from the EEPROM are transferred to the SitePlayer and hence the outside world.

The sequence is:

\begin{itemize}
	\item Power applied to WebBrick
	\item Internal web server  boots
	\item PIC boots waits 1-2 seconds for web server to settle
	\item PIC starts to transfer configuration to web server, this takes about 3 seconds
	\item If the PIC contains an IP address definition, this is set into the web server.  
        	Just before this point the WebBrick will have the 
		default address of 10.100.100.100.
	\item Operational Status set to 'Normal Operation' Interrupts enabled
	\item One Wire Bus searched for temperature sensors
	\item At 30 seconds after power up the internal real time clock is read, it is checked once a minute thereafter.
	\item Schedules are evaluated to see if any actions need to be executed.  i.e. 
		if the WebBrick is started at 10 o'clock and there is a scheduled action at 9 o'clock 
		this action will be executed.  Therefore if a WebBrick was controlling a central heating
		system it would recover from a power failure and carry on where it left off.
\end{itemize}

\subsection{WebBrick Heartbeat}

A LED connected to the PIC chip is used as a Heartbeat to show that the system is working.
It switches on and off at approx 1 second intervals with an equal off and on time once the 
WebBrick has completed its startup.
This LED is driven from the main program loop in the PIC chip.

If it stops flashing then the software has encountered an error, this would be a very dire state and if you ever see it
you should let us know.  We've only ever seen this happen with very noisy power supplies. 

During start-up the Heartbeat will flash quickly.  If the factory reset \index{factory reset} button is held in
at power-up the Heartbeat will remain on for two seconds before starting the load defaults and normal start-up operations.

[Future functionality] If the web interface is logged in then the LED will flash 
faster spending more time on and less time off, if the web home page has its controls disabled it will flash
slower spending more time off than on. [Not yet implemented]

