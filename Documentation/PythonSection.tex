\section{WebBrick and Python}

\subsection{Introduction}

Python is a very useful langauge. it can be used by almost anyone and allows for a full range of programming 
from simple scripting to full on object orientated programming. Webbrick Systems have used Python to implement 
the interface service between the webbrick and ITunes (Apple).

For the WebBrick we provide a number of Python resources, these fall into the following catergories:

\begin{tabular}{l|p{12cm}}

	Class Libraries&These allow the user to create programs that interact with WebBricks
				using routines that have been fully tested and debugged.  This allows
				the programmer to get on with building the automation they desire rather
				than getting bogged down in the details of network protocols.\\
	Utilities&These allow for common operatios such as uploading and downloading
				WebBrick configuration files.\\
	Example Code&This allows people to quickly build on some of the ideas we already had and
				implemented.\\
\end{tabular}

\subsection{Class Libraries}

\subsection{wb6.py the WebBrick class library}
\index{wb6.py}

This python module provides basic commands that can be sent to a WebBrick.

To import this library use the following in your code:

\begin{verbatim} import wb6 \end{verbatim}

From this point you can use this module to interact with WebBricks, for example
if you wanted to switch on a particular output channel you might use
somethine like

wb6.DigOn( '10.100.100.100', 3 )

\subsection{Functions implemented by wb class}

\subsubsection{getStatusXml}
getStatusXml(\param{adrs})

Retrieve the Xml blob that describes the current status of the addressed WebBrick
    \param{adrs} is the Ip address/Dns Name of the WebBrick being targetted.

\subsubsection{getConfigXml}
getConfigXml(\param{adrs})

Retrive the Xml blob that describes the current configuration of the addressed WebBrick
\param{adrs} is the Ip address/Dns Name of the WebBrick being targetted.

\subsubsection{DigTrigger}
DigTrigger(\param{adrs}, \param{chn}) 

Generate a digital trigger on a digital input

    \param{adrs} is the Ip address/Dns Name of the WebBrick being targetted.
    \param{chn} is the channel on the webBrick being targetted

\subsubsection{DigOn}
DigOn(\param{adrs}, \param{chn})

Set a digital On

    \param{adrs} is the Ip address/Dns Name of the WebBrick being targetted.
    \param{chn} is the channel on the webBrick being targetted

\subsubsection{DigOff}
DigOff(\param{adrs}, \param{chn})

Set a digital Off

    \param{adrs} is the Ip address/Dns Name of the WebBrick being targetted.
    \param{chn} is the channel on the webBrick being targetted

\subsubsection{DigToggle}
DigToggle(\param{adrs},\param{chn})

Toggle a digital Off

    \param{adrs} is the Ip address/Dns Name of the WebBrick being targetted.
    \param{chn} is the channel on the webBrick being targetted

\subsubsection{DigDwell}
DigDwell(\param{adrs},\param{chn},\param{DwellNr})

Set a digital channel on for Dwell Time

    \param{adrs} is the Ip address/Dns Name of the WebBrick being targetted.
    \param{chn} is the channel on the webBrick being targetted
    \param{DwellNr} is the dwell number

\subsubsection{AnOutSp}
AnOutSp(\param{adrs},\param{chn},\param{sp}) 

Set an analogue channel to preset sp.

        \param{adrs} is the Ip address/Dns Name of the WebBrick being targetted.
        \param{chn} is the channel on the webBrick being targetted
        \param{sp} is the preset number


\subsubsection{AnOutPercent}
AnOutPercent(\param{adrs},\param{chn},\param{val})

Set an analogue channel to a specific per-cent level.

    \param{adrs} is the Ip address/Dns Name of the WebBrick being targetted.
    \param{chn} is the channel on the webBrick being targetted
    \param{val} is a value 0-100 where 100 if 10V output

\subsubsection{Send}
Send(\param{adrs},\param{cmd})

Send a specific WebBrick command.

    \param{adrs} is the Ip address/Dns Name of the WebBrick being targetted.
    \param{cmd} is the command to be sent, see the section on Commands.

\subsubsection{GetXml}
result = GetXml(\param{adrs},\param{xmlName})

Retrieve a specific Xml blob from a webBrick.

    \param{adrs} is the Ip address/Dns Name of the WebBrick being targetted.
    \param{xmlName} The name of the Xml blob, currently xmlStatus.xml and xmlCfg.xml.
    \param{result} the return value is the Xml if the parameters are valid

\subsubsection{wbUdpEvents.py}
\index{wbUdpEvents.py}
This is a class that captures and delivers Udp events to an event target. To use this you create an event
target derived from wbUdpEvents.udpPacket for the events to be delivered to and pass this when you create 
wbUdpEvents.wbUdpEvents. 

wbMonitor is an example using this, See under Python Utilities.

\subsubsection{wbXmlEvent.py}
\index{wbXmlEvent.py}

This is a class that builds upon wbUdpEvents to turn the UDP events from a webbrick into an Xml format, another example of an
Event target for wbUdpEvents.  wbXmlEventTest.py (see under Python Utilities) uses this class.

\subsubsection{WebBrickConfig.py}
\index{WebBrickConfig.py}

This class contains methods to read a WebBrick XML configuration file, and convert them into WebBrick configuration commands.
There is a command interface to this class that uses it to save and restore configuration to/from disk. 
See under Python Utilities.

\subsubsection{wbStatus.py}
\index{wbStatus.py}

This class retrieves a copy of the current WebBrick Status that can then be asked for specific details. Note
the object reads and caches the values, so you get a current snapshot. To get updated values create a new object.
There is no point retrieveing tha values at more a 1 second interval and it is recommended that you do as much as possible
with a single retrieved snapshot.

\begin{enumerate}
\item wbStatus(\param{adrs}) Construct a wbStatus object.
        \param{adrs} is the Ip address/Dns Name of the WebBrick from which the status should be retrieved.

\item digOutState = digOutState(\param{chn}) return current status of a digital output.
         \param{chn} the digital output channel number.
         \param{result} One of True or False for On or Off.
\item digInState = digInState(\param{chn}) return current status of a digital input or monitor input (channels 8-11).
         \param{chn} the digital input channel number.
         \param{result} One of True or False for On or Off.
\item anOutVal = anOutVal(\param{chn}) return current setting of an analogue output channel.
         \param{chn} the analogue output channel number.
         \param{result} 0-100 where 100 is an output of 10V
\item anInVal = anInVal(\param{chn}) return current level of an analogue input channel.
         \param{chn} the analogue input channel number.
         \param{result} 0-100 where 100 is an input of 5V
\item tempVal = tempVal(\param{chn}) return current temperature.
         \param{chn} the analogue input channel number.
         \param{result} a signed temperature value multipled by 16, temperarure resolution is to 1/16 of a degree celcius.
\item getTime = getTime() return current time at webBrick.
         \param{result} The time as a formatted string.
\item getHour = getHour() return current hour at webBrick.
         \param{result} The hour in 24 hour format 0-23
\item getMinute = getMinute() return current minute at webBrick.
         \param{result} The minute as 0-59
\item getSecond= getSecond() return current second at webBrick.
         \param{result} The second as 0-59
\item getDay = getDay() return current day at webBrick.
         \param{result} The day as a number 0-6 for Sunday through to Saturday.
\item getVersion = getVersion() return firmware version for WebBrick.
         \param{result} String version number.
\item getOpState = getOpState() return operational state of webBrick.
         \param{result} See comnmand section for values.
\item getLoginState = getLoginState() return login level at webBrick.
         \param{result} 0-3.
\item getLastError = getLastError() return result of last command to webbrick.
         \param{result} 0 was no error.
\end{enumerate}

\subsubsection{wbCurrentCfg.py}
\index{wbCurrentCfg.py}
This class retrieves a copy of the current WebBrick configuration that can then be asked for specific details.
Not yet implemented.

\begin{enumerate}
\item wbCfg(\param{adrs}) Construct a wbCfg object
        \param{adrs} is the Ip address/Dns Name of the WebBrick from which the configuration should be retrieved.

\item NodeName = NodeName()Retrive node name
\item NodeNumber = NodeNumber()Retrive node number
    
\end{enumerate}

\subsection{Python Utilities}

\subsubsection{wbMonitor.py}
\index{wbMonitor.py}
Uses wbUdpEvents to display received webbrick events.

\subsubsection{wbXmlEventTest.py}
\index{wbXmlEventTest.py}
Uses wbXmlEvent to display received webbrick events as Xml blobs.

\subsubsection{WbCfg.py}
\index{WbCfg configuration download utility}

This program is used to configure a WebBrick from a configuration file and to save there configuration. It uses the 
class WebBrickConfig.

Usage wbCfg.py prog [options] \param{WbAdrs}

\begin{tabular}{l|l|p{12cm}}
    option&option parameters&Description\\
    \hline
    \hline
    -d,--display&&Display WebBrick configuration (default)\\
        \hline
    -s,--save&(\param{fileName})&save webbrick configuration\\
        &\param{fileName}&name of file to save configuration to.\\
        \hline
    -u,--update&(\param{fileName})&Update WebBrick configuration\\
        &\param{fileName}&name of file to read configuration from.\\
        \hline
    -p,--password&(\param{password})&Access password\\
        &\param{password}&Access password, default password.\\
        \hline
    -t,--trace&&Trace operation\\
        \hline
\end{tabular}

You should note that during configuration we set the operating state - to '1' (startup) to disable 
all the functions of the WebBrick.We do this to ensure that there are no extraneous actions or 
commands in progress that would interrupt or corrupt the configuration process.

The save format at present is the wbCfg.xml format.

\subsection{Python Examples}

There are some Python examples that you can use to generate the command packets that WebBricks will respond to.  

\subsubsection{wbDigOut.py}
\index{wbDigOut.py}

This program turns a single digital channel on off or toggles it.

usage:  python wbDigOut.py \param{ipAdrs} \param{channel} \param{action}

\subsubsection{wbExcercise.py}
\index{wbExcercise.py}

This program just repeatadly toggles all the digital channel in turn. If you have LEDs attached to the outputs you will
get a NightRider type scan.

