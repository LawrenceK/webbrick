\section{Hardware Details}

Version 4.2 of the PCB uses the following parts list:


    \begin{description}
    \item [IC1] PIC18F452-I/P (see Gotchas for note on Silicon revision levels)
    \item [IC2] 7805 5V Reg
    \item [IC3] LM358 dual OP Amp, buffers and doubles PWM outputs to give 0-10V
    \item [IC4] 2803A output buffer for 12V 500mA outputs
    \item [Xtal] 20Mhz (could be 40Mhz later)
    \item [Pulse Transformer]
    \item [SitePlayer] See www.siteplayer.com for details
    \end{description}

The following connectors and connections are used:

  \index{Connections}

    \begin{description}
    \item [DIN] Push button inputs, note that the PIC's internal pull-ups are ENABLED
    \item [DOL] Digital Outputs TTL level
    \item [DOH] Digital Outputs 12V 500mA max
    \item [JP4] Rotary Encoder pin 1 'Up' or A, pin 2 'Down' or B.
    \item [JP1 1] Analogue Input 0-5V
    \item [JP1 2-5]  Monitor Input TTL Level     
    \item [JP1 7] One Wire Bus Data \(See Also OWB below\)
    \item [JP1 8] DMX Passthru out TTL Level \(needs MAX232 chip or direct connection to Milford Instruments board\)
    \item [X2 1] Gnd for Analogue output
    \item [X2 2,3] Analogue Outputs
    \item [OWB] One Wire Bus
    \item [V+] A useful source of +5V
    \end{description}


\begin{figure}[H]
\centering
\includegraphics[width=0.7\textwidth]{Images/WBConnections.jpg}
\caption{WebBrick Connections}
\end{figure}


Note that since version 4.2 of the PCB, pull-ups for the rotary encoder are included.


