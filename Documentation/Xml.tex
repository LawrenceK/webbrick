\section {Xml Structure}
\index{XML}

A WebBrick can serve up two Xml files, one of these (WbStatus.xml) provides the current status of dynamic information, i.e.
current output state and the other (WBCfg.xml) contains the complete WebBrick configuration.

The element names are chosen to identify the commands used to modify attributes within the PIC chip.

The XML is intended to be consumed by other computer software more than used to display to the end user in its raw form. If
you try and look at it in a normal text editor you will find some very long line lengths.

\subsection{Trigger Encoding,XML}
    When a trigger sequence is sent out from the webbrick it is encoded into 3 bytes encoded as 
    follows.

    \begin{enumerate}
    \item[ConfigByte1]
    This uses the following Bit fields
        \begin{enumerate}
        \item[Bits 0-4] The encoding of these bits depends on the command group.
        \item[Bits 5-6] These bits select a command group.
            \begin{enumerate}
            \item[0] General commands
            \item[1] Not currently used
            \item[2] Not currently used
            \item[3] Dwell commands.
            \end{enumerate}
        \item[Bits 7] This indicates whether to send a UDP packet.
            \begin{enumerate}
            \item[0] No UDP packet will be sent in response to a input trigger
            \item[1] General UDP Packet, see packet contents for further details.
            \end{enumerate}
        \end{enumerate}

        General Commands - these use 5 bits as a command or action.
            \begin{enumerate}
            \item[0] No Action
            \item[1] Off
            \item[2] On
            \item[3] Momentary - short duration
            \item[4] Toggle
            \item[5] Unused
            \item[6] Unused
            \item[7] Next
            \item[8] Prev
            \item[9] Set Low Threshold
            \item[10] Set High Threshold
            \item[11] Adjust Low Threshold
            \item[12] Adjust High Threshold
            \item[13] Send Infra Red
            \item[14] Up
            \item[15] Down
            \item[16] Set DMX channel
            \item[17] Count
            \end{enumerate}

        Dwell Commands - these use 2 bits as a command and 3 bits to select a dwell number.

            \begin{enumerate}
            \item[0-2] Dwell number or index
            \item[3-4] Dwell command
            \end{enumerate}
            
            \begin{enumerate}
            \item[0 - 00] Dwell On
            \item[1 - 16] Dwell Off
            \item[2 - 32] Dwell Cancel
            \item[3 - 48] Dwell Always
            \end{enumerate}

    \item[ConfigByte2]

    This uses the following bit fields.
        \begin{enumerate}
        \item[Bit 7] If this bit is set then the target channel is analogue and the following analogue 
            format applies.
        \end{enumerate}
        Analogue format.
        \begin{enumerate}
        \item[Bit 7] bit is set
        \item[Bits 0-3] This is the setpoint to be used in analogue commands
        \item[Bits 4-6] This is the analogue channel on which the action is to be performed.
        \end{enumerate}
        Non analogue format.
        \begin{enumerate}
        \item[Bit 7] bit is clear
        \item[Bit 6] If Clear then it is a digital channel, if set then it is a Scene.
        \item[Bits 0-3] This is the digital channel or scene number.
        \item[Bits 4-5] Reserved for other possible options, keep as 0.
        \end{enumerate}

    \item[ConfigByte3]
    This is the remote node to be triggered for remote actions. It will be set in all
    UDP packets so can be used for other purposes when type is General or Alarm.

    \item[ConfigByte4]
    This is a reserved byte not yet used.
    \end{enumerate}

\subsection {WbStatus.xml}
\index{WbStatus.xml}

    \begin{tabular}{l|l}
    Element&Description\\
    \hline
    WebbrickStatus&Overall Element in WbStatus.xml\\
    SN&Node number\\
    Error&Response to last command\\
    Context&Operational State\\
    LoginState&Login level\\
    DI&Digital input state\\
    DO&Digital output state\\
    Clock&Current webbrick clock\\
    Date&Not implemented\\
    Time&Current webbrick time\\
    Day&Current webbrick day of week\\
    OWbus&One wire bus status\\
    Tmps&Temperature elements\\
    Tmp&Temperature id nn values\\
    AOs&Analogue output elements\\
    AO&Analogue output id nn values\\
    AIs&Analogue input elements\\
    AI&Analogue input id nn values\\
    \end{tabular}

For example of using this see wb6Status.py in the python API or wb6Status.php in the PHP API.

\subsubsection {Error}
The element contents are 0 or an error code if the last webbrick command was invalid.

    \begin{tabular}{l|l}
    Value&Description\\
    \hline
    0&No Error\\
	1&Bad Command\\
	2&Bad Parameter\\
	3&Not Logged In\\
	4&Address Locked, attempt to change IP address on address locked webbrick.\\
	5&In Startup\\
	6&No Command yet issued\\
    \end{tabular}

\subsubsection {Context}
Operating context.

    \begin{tabular}{l|l}
    Value&Description\\
    \hline
    0&Startup\\
    \end{tabular}

\subsubsection {LoginState}
Login state.

    \begin{tabular}{l|l}
    Value&Description\\
    \hline
    0&Logged out\\
    1&controls enabled\\
    2&configuration\\
    3&installer\\
    \end{tabular}

\subsubsection {DI}
Digital input state, bit mapped integer where bit 0 is the current state of digital input 0 (when using a zero base). 

\subsubsection {DO}
Digital output state, bit mapped integer where bit 0 is the current state of digital output 0 (when using a zero base). 

\subsubsection {Clock}
Contains the current time as known by the WebBrick, the XML includes full date but this is not implemented within the
system at present. The element Date within Clock contains date string and Time contains the time in 24 hour format.
	
\subsubsection {OWBus}
Contains the status of the one wire bus. The value of 255 shows no sensors detected and a small integer is a bit mapped 
value with a bit for each temperature sensor.

\subsubsection {Tmps}
Contains one entry for each possible temperature sensor.

\subsubsection {Tmp}
Contains an id attribute, the current low threshold and high thresholds as attributes and the 
current temperature reading. Note these are in 16ths of a degree so convert to float/real number and 
divide by 16 to get value.

\subsubsection {AO}
Contains the current analogue output value for an analogue output channel, this is a number from 0-100 where 100
is 10V output.

\subsubsection {AI}
Contains an id attribute, the current low threshold and high thresholds as attributes and the 
current analogue reading. Note these are in the range 0 - 100 where 100 is 5V at the terminal.

\subsubsection {Sample wbStatus.xml}
The following is a sample copy of a webbrick Xml status file.

\verbatiminput{WbStatus.xml}

\subsection {WbCfg.xml}
\index{WbCfg.xml}

    \begin{tabular}{l|l}
    Element&Description\\
    \hline
    WebbrickConfig&Overall Element in WbStatus.xml\\
    NN&Node Name\\
    SN&Node number\\
    SRs&Rotary encoder elements\\
    SR&Rotary encoder step value\\
    IR&Infra red address received by webbrick on IR reception.\\
    SF&Analogue fade rate\\
    CDs&digital input elements\\
    CD&digital id nn input\\
    Trg&Trigger setting\\
    TrgL&Low threshold and trigger setting\\
    TrgH&High threshold and trigger setting\\
    CCs&Scene elements\\
    CC&Scene id nn settings\\
    CWs&Dwell elements\\
    CW&Dwell id nn setting\\
    CSs&Set point elements\\
    CS&Set point id nn \\
    CTs&Temperature elements\\
    CT&Temperature input id nn value,name,thresholds\\
    CIs&Analogue input elements\\
    CI&Analogue input id nn value,name,thresholds\\
    CEs&Scheduled event elements\\
    CE&Scheduled event id nn \\
    NOs&Digital output elements\\
    NO&Digital output id nn name \\
    NMs&Monitor input elements\\
    NAs&Analogue output elements\\
    NA&Analogue output id nn name \\
    MM&Mimic configuration\\
    \end{tabular}

\subsubsection {WebbrickConfig}
Contains the firmware version number

\subsubsection {NN}
Contains the configured node name.

\subsubsection {SI}
Contains the IP address and MAC address of the webbrick.

\subsubsection {SN}
Contains the configured webbrick node number.

\subsubsection {SR}
Contains attributes id and the Value for the rotary step value.

\subsubsection {SF}
Contains the fade rate counter, this controls how fast the analogue channels fade up and down.

\subsubsection {CDs}
Container element for all the digital input channel elements.

\subsubsection {CD}
Configuration details for a digital input. Contains an id, the name and the options for the input and a nested trigger
element. See start of section for trigger details.

\subsubsection {CCs}
Container element for all the scene elements.

\subsubsection {CS}
Configuration details for a scene. Contains an id attribute and controls for the digital outputs and analogue outputs.
The Dm attribute identifies which digital channels are affected by this scene and Ds for those channels whether to 
switch the channel off or on. Am identifies which analogue channels are affected and Av the set points for those 
channels. Note when a scene is selected by a trigger the trigger action will override the On action here, so that 
a scene can be dwelled etc. If the scene says Off then Off it goes.

\subsubsection {CWs}
Container element for all the dwell values.

\subsubsection {CW}
Contain the configured dwell values in seconds.

\subsubsection {CSs}
Container element for all the analogue set point values.

\subsubsection {CS}
Contains each of the set point values in the range 0-100.

\subsubsection {CTs}
Container element for all the temperature sensor configuration.

\subsubsection {CT}
Contains an id and a name for a temperature sensor and an embedded trigger along with the threshold
at which it occurs. TrgL is the low threshold amd TrgH is the high threshold for the sensor. Threshold
values are in 16ths of a degree.

\subsubsection {CIs}
Container element for all the analogue input configuration.

\subsubsection {CI}
Contains an id and a name for an analogue channel and embedded triggers along with the threshold
at which they occurs. TrgL is the low threshold amd TrgH is the high threshold for the sensor. Threshold
values are in the range 0-100.

\subsubsection {CEs}
Container element for all the scheduled event configuration.

\subsubsection {CE}
Each element contains an id, Days, Hours, Minutes attribute. The Days attribute is bit mapped for days 0-6
and identifies on which days of the week the event occurs. The embedded Trg element is the action to perform.

\subsubsection {NOs}
Container element for all the digital output elements.

\subsubsection {NO}
Each element contains an id and a name for a digital output.

\subsubsection {NAs}
Container element for all the analogue output elements.

\subsubsection {NA}
Each element contains an id and a name for a analogue output.

\subsubsection {IR}
Contains the Infra red RC5 address recognised by the webbrick.

\subsubsection {MM}
Contains the current mimic settings. The 'lo' and 'hi' attributes are the off an on mimic settings, this value
is in the range 0-63. The 'dig' attribute is a 32 bit number with 4 bits for each digital output that maps the 
output channel to a mimic, as there are only 8 mimics a channel number greater than 7 is basically No Mimic. 
The 'an' attribute provides the same mapping for the analogue outputs. Finally the 'fr' attribute controls
how fast the mimics fade from one setting to the next.

\subsubsection {Sample wbCfg.xml}
The following is a sample copy of a webbrick Xml config file.

\verbatiminput{WbCfg.xml}
