\section{WebBrick and PHP}

\subsection{Introduction}

PHP is a web application langauge that allows you to mix business logic and user interface logic in a single file, this file
is processed on a web server to mix fixed and variable parts of the user interface. http://www.php.net

For the WebBrick we provide a number of PHP resources, these fall into the following catergories:

\begin{tabular}{l|p{12cm}}
    \hline
	Libraries&These allow the user to create programs that interact with WebBricks
				using routines that have been fully tested and debugged.  This allows
				the programmer to get on with building the user interface they desire rather
				than getting bogged down in the details of network protocols.\\
    \hline
	Example Code&This allows people to quickly build on some of the ideas we already had and
				implemented.\\
\end{tabular}

\subsection{Libraries}

\subsubsection{wb.php basic comms libray}
\index{wb.php}

This php module provides basic communication with the WebBrick and is mainly intended to be a building block.

To use this library use the following in your code:

\begin{verbatim} include("../API/wb.php") ;\end{verbatim}

From this point you can use this module to interact with WebBricks, for example
if you wanted to switch on a particular output channel you might use
somethine like

wbSendCmd( '10.100.100.100', 'DO3;N' )

\subsubsection{wb6.php the WebBrick 6 library}
\index{wb6.php}

This module provides basic commands that can be sent to a WebBrick.

To import this library use the following in your code:

\begin{verbatim} include("../API/wb6.php") ;\end{verbatim}

From this point you can use this module to interact with WebBricks, for example
if you wanted to switch on a particular output channel you might use
somethine like

wb6DigOn( '10.100.100.100', 3 )

\subsubsection{Functions implemented by wb6}

\begin{tabular}{l|p{12cm}}
    function&Description\\
    \hline
    \hline
    wb6Trigger(\param{adrs}, \param{chn})&Generate a digital trigger on a digital input\\
        \param{adrs}&is the Ip address/Dns Name of the WebBrick being targetted.\\
        \param{chn}&is the channel on the webBrick being targetted\\
        \hline
    wb6DigOn(\param{adrs}, \param{chn})&Set a digital On\\
        \param{adrs}&is the Ip address/Dns Name of the WebBrick being targetted.\\
        \param{chn}&is the channel on the webBrick being targetted\\
        \hline
    wb6DigOff(\param{adrs}, \param{chn})&Set a digital Off\\
        \param{adrs}&is the Ip address/Dns Name of the WebBrick being targetted.\\
        \param{chn}&is the channel on the webBrick being targetted\\
        \hline
    wb6DigToggle(\param{adrs},\param{chn})&Toggle a digital Off\\
        \param{adrs}&is the Ip address/Dns Name of the WebBrick being targetted.\\
        \param{chn}&is the channel on the webBrick being targetted\\
        \hline
    wb6DigDwell(\param{adrs},\param{chn},\param{DwellNr})&Set a digital channel on for Dwell Time\\
        \param{adrs}&is the Ip address/Dns Name of the WebBrick being targetted.\\
        \param{chn}&is the channel on the webBrick being targetted\\
        \param{DwellNr}&is the dwell number\\
        \hline
    wb6SetPoint(\param{adrs},\param{chn},\param{sp})&Set an analogue channel to preset sp.\\
        \param{adrs}&is the Ip address/Dns Name of the WebBrick being targetted.\\
        \param{chn}&is the channel on the webBrick being targetted\\
        \param{sp}&is the preset number\\
        \hline
    wb6AnOutValue(\param{adrs},\param{chn},\param{val})&Set an analogue channel to a specific per-cent level.\\
        \param{adrs}&is the Ip address/Dns Name of the WebBrick being targetted.\\
        \param{chn}&is the channel on the webBrick being targetted\\
        \param{val}&is a value 0-100 where 100 if 10V output\\
        \hline
    getStatusXml(\param{adrs})&Retrive the Xml blob that describes the current status of the addressed WebBrick\\
        \param{adrs}&is the Ip address/Dns Name of the WebBrick being targetted.\\
        \hline
    getConfigXml(\param{adrs})&Retrive the Xml blob that describes the current configuration of the addressed WebBrick\\
        \param{adrs}&is the Ip address/Dns Name of the WebBrick being targetted.\\
        \hline
\end{tabular}

\subsubsection{wb6Status.php}
\index{wb6Status.php}

This class retrieves a copy of the current WebBrick Status that can then be asked for specific details. Note
the object reads and caches the values, so you get a current snapshot. To get updated values use the load method.
There is no point retrieveing tha values at more a 1 second interval and it is recommended that you do as much as possible
with a single retrieved snapshot.

\begin{tabular}{l|p{12cm}}
    call&Description\\
    \hline
    wbStatus()&Construct a wbStatus object.\\
        \hline
    load(\param{adrs})&load a wbStatus object with a status snapshot.\\
        \param{adrs}&is the Ip address/Dns Name of the WebBrick from which the status should be retrieved.\\
        \hline
    getOneWireBus = getOneWireBus(\param{chn})&return current status of the one wire bus.\\
        \param{result}&Bit map showing which temperature sensors are available.\\
        \hline
    tempVal = getTemp(\param{chn})&return current setting of an analogue output channel.\\
        \param{chn}&the temperature sensor number.\\
        \param{result}&temperature.\\
        \hline
    digOutState = digOutState(\param{chn})&return current temperature.\\
        \param{chn}&the analogue output channel number.\\
        \param{result}&0-100 where 100 is an output of 10V\\
        \hline    
    anOutVal = getAnOut(\param{chn})&return current setting of an analogue output channel.\\
        \param{chn}&the analogue output channel number.\\
        \param{result}&0-100 where 100 is an output of 10V\\
        \hline
    anInVal = getAnIn(\param{chn})&return current level of an analogue input channel.\\
        \param{chn}&the analogue input channel number.\\
        \param{result}&0-100 where 100 is an input of 5V\\
        \hline
    digInState = getDigIn(\param{chn})&return current status of a digital input or monitor input (channels 8-11).\\
        \param{chn}&the digital input channel number.\\
        \param{result}&One of True or False for On or Off.\\
        \hline
    digOutState = getDigOut(\param{chn})&return current status of a digital output.\\
        \param{chn}&the digital output channel number.\\
        \param{result}&One of True or False for On or Off.\\
        \hline
    monitorInState = getMonitor(\param{chn})&return current status of a monitor input.\\
        \param{chn}&the monitor channel number.\\
        \param{result}&One of True or False for On or Off.\\
        \hline
    error = getCmdError()&return result of last command to webbrick.\\
        \param{result}&0 was no error.\\
        \hline
    state = getOperationalState()&return operational state of webBrick.\\
        \param{result}&See comnmand section for values.\\
        \hline
    state = getLoginState()&return login level at webBrick.\\
        \param{result}&0-3.\\
        \hline
    date = getWbDate()&return current date at webBrick.\\
        \param{result}&The date as a formatted string, not yet implemented at webbrick.\\
        \hline
    getTime = getTime()&return current time at webBrick.\\
        \param{result}&The time as a formatted string.\\
        \hline
    getDay = getDay()&return current day at webBrick.\\
        \param{result}&The day as a number 0-6 for Sunday through to Saturday.\\
        \hline
\end{tabular}

\subsubsection{wb6Cfg.php}
\index{wb6Cfg.php}
This class retrieves a copy of the current WebBrick configuration that can then be asked for specific details.

\begin{tabular}{l|p{12cm}}
    function&Description\\
    \hline
    wbConfig()&Construct a wbConfig object\\
        \hline
    load(\param{adrs})&load a wbCfg object\\
        \param{adrs}&is the Ip address/Dns Name of the WebBrick from which the configuration should be retrieved.\\
        \hline
    NodeName = getNodeName()&Retrive node name\\
        \param{result}&is the node name.\\
        \hline
    NodeNumber = getNodeNumber()&Retrive node number\\
        \param{result}&is the node number.\\
        \hline
    fadeRate = getFadeRate()&Retrive fade rate\\
        \param{result}&is the fade rate.\\
        \hline
    ipAddress = getIpAddress()&Retrive IP address\\
        \param{result}&is the IP address.\\
        \hline
    macAddress = getMacAddress()&Retrive ethernet MAC address\\
        \param{result}&is the ethernet MAC address.\\
        \hline
    rotaryStep = getRotary( chan )&Retrieve rotary step value\\
        \param{chan}&rotary channel number.\\
        \param{result}&is the step value.\\
        \hline
    dwell = getDwell( idx )&Retrieve dwell value\\
        \param{idx}&dwell number.\\
        \param{result}&dwell in seconds.\\
        \hline
    dwellStr = getDwellStr( idx )&Retrieve dwell value\\
        \param{idx}&dwell number.\\
        \param{result}&dwell as a display string.\\
        \hline
    sp = getSetPoint( idx )&Retrieve set point value\\
        \param{idx}&set point number.\\
        \param{result}&setpoint 0-100\%.\\
        \hline
    nameStr = getDigOutName( chan )&Retrieve name for digital output\\
        \param{chan}&output number.\\
        \param{result}&name string.\\
        \hline
    nameStr = getMonitorName( chan )&Retrieve name for monitor input\\
        \param{chan}&monitor number.\\
        \param{result}&name string.\\
        \hline
    nameStr = getAnalogueOutName( chan )&Retrieve name for analogue output\\
        \param{chan}&analogue number.\\
        \param{result}&name string.\\
        \hline
    array = getDigInTrigger( chan )&Retrieve values for digital trigger\\
        \param{chan}&digital in number.\\
        \param{result}&associative array.\\
        \hline
    array = getTempTriggerLow( chan )&Retrieve values for low temperature trigger\\
        \param{chan}&temperature number.\\
        \param{result}&associative array.\\
        \hline
    array = getTempTriggerHigh( chan )&Retrieve values for high temperature trigger\\
        \param{chan}&temperature number.\\
        \param{result}&associative array.\\
        \hline
    array = getAnalogueTriggerLow( chan )&Retrieve values for low analogue trigger\\
        \param{chan}&analogue in number.\\
        \param{result}&associative array.\\
        \hline
    array = getAnalogueTriggerHigh( chan )&Retrieve values for high analogue trigger\\
        \param{chan}&analogue in number.\\
        \param{result}&associative array.\\
        \hline
    array = getScheduledEvent( chan )&Retrieve values for a scheduled event trigger\\
        \param{chan}&analogue in number.\\
        \param{result}&associative array.\\
        \hline
    array = getScene( chan )&Retrieve values for a scene\\
        \param{chan}&scene number.\\
        \param{result}&associative array.\\
        \hline
\end{tabular}

Trigger results are returned as an asociative array with some of the following parameters in it, refer to commands 
section for details on these values.

\begin{tabular}{l|p{12cm}}
    attribute name&Description\\
    \hline
    Name&channel name\\
    Value&threshold value\\
    Hours&scheduled time\\
    Minutes&scheduled time\\
    Days&scheduled days\\
    actionNr&action type number\\
    action&action type string\\
    dwell&dwell number\\
    UDPRemNr&UDP packet type number\\
    UDPRem&UDP packet type string\\
    RemNode&target remote node.\\
    typeNr&channel type number\\
    type&channel type string\\
    setPoint&setpoint number\\
    pairChn&target channel number\\
\end{tabular}

Scene results are returned as an asociative array with the following parameters in it.

\begin{tabular}{l|p{12cm}}
    attribute name&Description\\
    \hline
    Digital\param{nn}&Ignore|On|Off\\
    Analogue\param{nn}&Ignore|SetPoint\param{mm}\\
\end{tabular}
where \param{nn} is a channel number and \param{mm} is a setpoint number.

\subsection{PHP Examples}

There is a PHP example that you can modify to produce your own central user interface.  

\subsubsection{PanelLib.php}
\index{PanelLib.php}

This provides a library of code for a user interface panel that is capable of displaying details from one or more 
webBricks and issuing commands to them. This uses an Xml file standard.xml to define a user interface that consists 
of a number of columns that contain status information and/or commands.

The xml file has a top level element called status with a single attribute cols that specifies the number of columns on the
display. There are then a set of elements called item. Each item has a Type element and typically a channel number and an
IP address to identify the webbrick and the data to access from it. Each entry may also contain a Trig element that is used to
specify a command to be issued.

item Type may be one of:

\begin{tabular}{l|p{12cm}}
    type&Description\\
    \hline
    Header&generates a text of Cols width.\\
    Temp&displays a temperature.\\
    inAnalogue&displays an analogue input value.\\
    Analogue&displays an analogue output.\\
    Switch&displays a digital output state.\\
    Monitor&displays a monitor state.\\
\end{tabular}

\subsubsection{standard.php}
\index{standard.php}

This is the start page for the standard panel.

