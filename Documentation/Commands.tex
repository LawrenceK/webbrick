\section {Command Structure}
\index{command structure}

The WebBrick has a command interface that accepts command strings, these can be generated
by the Web interface and can also be sent to the WebBrick over the network. For network delivery
the commands can be sent using an HTTP URL with the command encoded in the parameters or a UDP transmission,
the HTTP use is preferable as you get an indication of whether the command was received at the WebBrick.

\subsection{General}
\index{Command Structure, General}
All commands start with a 2 character identifier, and are followed by any required parameters and terminated 
by a colon (':') character after the parameters, 
which means end-command. Each parameter to a command is terminated by a semi-colon (;). In this document 
items in \param{is a user parameter} 
and should be replaced with real world values. Items in [...] are optional parameters, therefore are not essential to replace. Generally the first character within the 2 character 
identifier identifies the command group and the second identifies the entity type. No extra blank spaces should
be inserted into the command as unexpected things may happen or the command be rejected.

These commands can be sent to the WebBrick in two ways:

\begin{enumerate}
	\item By encoding the parameters into 
	an HTTP URL and accessing that URL
	\item or by sending a UDP packet to the WebBrick with a 
	correct Siteplayer header on it for delivery to the PIC chip. 
\end{enumerate}

The embedded UI uses the 
HTTP URL approach and it is recommended that this approach is used for external systems. If you want
to use UDP packets look at the python code in wbUdp.py for details.

The HTTP URL to be used is http://\param{ip Address}/cfg.spi?com=\param{commandString}
Where \param{ip address} is the WebBricks IP address or host name.
and \param{commandString} is one of the command strings documented here. Note some commands will
need you logged in for them to be processed.

To make it easier to understand the command strings the first character is generally used as follows:

\subsubsection{command groups}
\begin{tabular}{l|p{12cm}}
1st Letter&Command Group\\
\hline
C&Configure\\
S&Set\\
N&Name\\
D&Digital\\
A&Analogue\\
I&Infra Red\\
T&Thresholds\\
\end{tabular}

And the second character generally identifies the target channel type, as follows:
\subsubsection{entity type}
\begin{tabular}{l|p{12cm}}
2nd Letter&entity type\\
\hline
D&Digital input\\
A&Analogue Output\\
I&Analogue input\\
O&Digital Output\\
T&Temperature\\
C&Scene\\
R&Serial comms or rotary encoder\\
\end{tabular}

\subsubsection{Trigger Configuration}
\index{Trigger!configuration sequence}
\index{Trigger sequence}
A lot of the commands take a sequence of parameters referred to as a Trigger sequence, this 
defines what is to happen when a trigger event occurs, for example a digital input or 
a temperature threshold being crossed. 

The sequence of parameters is:

\param{A\textbar{}D\textbar{}S\textbar{}T\textbar{}I}\param{targetChn}; \param{SetPointNr}; \param{actionType}; \param{DwellNr}; \param{UDPType}; \param{AssociatedValue} [; \param{OptValue}]:

Where:

\begin{tabular}{l|p{12cm}}
Parameter&Description\\
\hline
\param{A\textbar{}D\textbar{}S\textbar{}T\textbar{}I}&is a single character that identifies whether the target channel is an 
analogue output(A), a digital output (D), a Scene (S), a Temperature sensor (T), or Infra Red Send\\
\param{targetChn}&is the channel number or Scene being targeted\\
\param{actionType}&is one of the values from the Action table (range 0-15) In the case of a Scene this
is used as the action for all analogue channels that are not marked to be ignored and for digital 
channels marked for On in the scene configuration.\\
\param{DwellNr}&is a dwell number and is ignored if the actionType is not a dwell. Note for DT command dwell 
is passed in seconds and not as a dwell index.\\
\param{SetPointNr}&is a setpoint number and is ignored if the target channel is not analogue, Note a Scene is
not an analogue channel and the SetPoint is taken from the Scene.\\
\param{UDPDo}&identifies whether a UDP packet type is to be sent\\
\param{AssociatedValue}&is a number associated with the action.\\
\param{OptValue}&is an optional reserved value.\\
\end{tabular}

\subsubsection{Actions}
Each trigger can cause one of the following actions to be performed on an output channel.

\begin{tabular}{l|l|p{12cm}}
Nr&Action&Description\\
\hline

0&None&no action\\
1&Off&switch off\\
2&On&switch digital channels on or an analogue channel to a setpoint.\\
3&Momentary&Switch a channel on for a small time period circa 200mS.\\
4&Toggle&Turn an on channel off and an off channel on. For analogue channels if the current setting is greater than 0 it is deemed to be on.\\
5&Dwell Always&Always Switch a channel on for a configured time period, if lamp already on this will switch off after a time period.\\
6&Dwell-Cancel&If a channel is Off Switch the channel on for a configured time period. 
If the channel is already On then switch the channel Off immediately.\\
7&Next&Move a channel to its next higher state, for analogue channels this is to the next set point, for digital channels this is equivalent to a toggle. If the target is any Scene change the current scene at the web brick up by one.\\
8&Prev&Reverse of Next, Down Setpoint, Toggle, Previous Scene.\\
9&SetLowThreshold&Change the low threshold for one of the analogue input or temperature sensor\\
10&SetHighThreshold&Change the high threshold for one of the analogue input or temperature sensor\\
11&AdjustLowThreshold&Move the low threshold for one of the analogue input or temperature sensor\\
12&AdjustHighThreshold&Move the high threshold for one of the analogue input or temperature sensor\\
13&SendIR&Send a command over the Infra Red emitter, RC5 only, the target RC5 channel is the command code 
and associated value is the RC5 address.\\
14&Up&Take a channel up a step, generally used for analogue outputs.\\
15&Down&Take a channel down a step, generally used for analogue outputs.\\
16&Set Dmx&Set a DMX channel to a level.\\
17&Count&Sets the channel to count mode.\\
18&Dwell On&Switch a channel on for a configured time period if it is not already on.\\
19&Dwell Off&Switch a channel off after a configured time period.\\
\end{tabular}

Notes
\begin{itemize}
\item Toggle, On and Off actions all override a current Dwell command.
\item A Dwell command issued during a current Dwell period will reset the Dwell time.
\item A Dwell Cancel command issued during a current Dwell period will end the dwell. A Dwell Cancel issued 
outside a Dwell period will switch the output on for the Dwell time.
\item At the end of an analogue dwell the output will return to the level prior to the dwell, this 
is useful for setting lights
to a low level and bringing them high on some trigger, e.g. security PIR.
\item Channel numbers are zero based for the internal commands.
\item Scenes are in two banks, one bank of 8 and one bank of 4, when performing Next and Prev Scene the scene change 
stays within the bank that the trigger definition identifies by selecting one of the scenes within the chosen bank.
\item String parameters cannot contain any of the characters '\textless{}\textgreater{}\&
\item Analogue channels go up and down setpoints with Next/Prev and step the level the rotary step value with Up/Down \&
\item Some commands are meaningless with some outputs, where possible reasonable actions are chosen. i.e. for digital 
channels Next/Prev/Up/Down are treated as a toggle command\\
\end{itemize}

\subsection{Error Codes}
These may be displayed in the header of the webbpage and may appear in the xml status.

    \begin{tabular}{l|l}
    Value&Description\\
    \hline
    0&No Error.\\
	1&Bad Command.\\
	2&Bad Parameter.\\
	3&Not Logged In.\\
	4&Address Locked, attempt to change IP address on address locked webbrick.\\
	5&In Startup state.\\
	6&No Command yet issued, post startup state.\\
    \end{tabular}
	
\subsection{Command summary}

\subsubsection{Static configuration commands}

\begin{tabular}{l|p{12cm}}
Command&Function\\
\hline
ND&Name digital Input\\
NO&Name digital output\\
NI&Name analogue input\\
NA&Name analogue output\\
NT&Name temperature sensor\\
NN&Name node\\
CD&Configure digital input\\
CI&Configure analogue input\\
CT&Configure temperature input\\
CS&Configure set point\\
CW&Configure dwell\\
CR&Configure serial interface.\\
CE&Configure scheduled event\\
CC&Configure a scene\\
SN&Set node number\\
SF&Set fade rate\\
SM&Set mimic brightness levels for on and off signals, and fade rate\\
CM&Configure mimic channels for analogue and/or digital outputs\\
ST&Set Time\\
SD&Set Date\\
SR&Set rotary encoder step\\
SI&Set Internet address\\
SA&Set IP address with verified MAC address; the !WebBrick sets its IP address to the specified value only if its 48-bit MAC address matches that supplied.  This command can be used with UDP broadcasts to set IP addresses when several WebBricks on a network have the same initial IP address.\\
SP&Set password for indicated security level\\
SO&Set option on (1) or Off (0) (Use with care: currently deprecated, and not reflected in XML configuration)\\
IR&Enable infrared receive\\
IT&Enable infrared transmit\\
IA&Sets the RC5 Infrared address for receiving\\
% Private command.
%ZO&Security option setting (e.g. IP address locking)\\
\end{tabular}

\subsubsection{Dynamic status setting commands}

\begin{tabular}{l|p{12cm}}
Command&Function\\
\hline
DI&Trigger digital input\\
DO&Set digital output\\
DT&Invoke a trigger event action using the supplied parameters. The parameters are a full trigger definition as can
be attached a local event source, i.e. digital input, but are acted upon when received. Note dwell is passed in seconds and not as a dwell index.\\
DM&Set one or more mimic outputs to specified levels\\
DA&Requests web brick to send a small number of UDP attention packets within a few seconds (currently: one immediately, and one more within a second).  This is used for WebBrick discovery.  (cf. factory reset causes attention packets to be sent every few seconds for a minute.)\\
AA&Set analogue output to one of the setpoints or to an absolute level.\\
SC&Set scene\\
SS&Set operational state\\
TA&Modify dynamic threshold for analogue input\\
TT&Modify dynamic threshold for temperature sensor\\
\end{tabular}


\subsubsection{Miscellaneous WebBrick control commands}

\begin{tabular}{l|p{12cm}}
Command&Function\\
\hline
LG&Login to security level associated with password - also logout.\\
RT&Re scan 1-wire bus for new devices\\
RU&Refresh web user interface data\\
RS&Reboot Siteplayer, PIC chip delays a while and then rebuilds the UI data.\\
RB&Reboot PIC\\
RI&Set RS485 driver off, input mode\\
RO&Set RS485 driver on, output mode\\
RD&Send serial data\\
IS&Sends an RC5 infrared command (channel) over a configured infrared emitter\\
\end{tabular}


\subsection{Commands}

All command are listed here.

\subsubsection{AA - Analogue output control}
\index{Analogue output control}
\index{Analogue Output ! control sequence}
AA\param{chn};[S\param{nn}\textbar{}\param{nn}]  Set Analogue output setpoint or absolute

Set an analogue output to a specific level.
chn is the analogue output channel number. If the output value is S\param{nn} then \param{nn} is the setpoint,
otherwise \param{nn} is the output level as an absolute value of between 0-100\%. 

e.g.
AA1;S1:

\subsubsection{CC - Configure Scene}
\index{Configure Scene}
CC\param{nr};[NFI][NFI][NFI][NFI][NFI][NFI][NFI][NFI]; [I\textbar{}S\param{nn}]; [I\textbar{}S\param{nn}]; [I\textbar{}S\param{nn}]; [I\textbar{}S\param{nn}]:

Configure a scene. A scene consists of optional settings for all the digital and analogue channels. Any channel
may be marked as Ignore (do not change). Digital channels also be marked as On or Off, whilst analogue channels
can be given a set point. The first group [NFI] is for the digital channels and is repeated as many times 
as required to cover the digital channels where a change is required, i.e. if you only want to modify 
digital channels 0, 1, 2 then you only need list 3 entries. If you want to only affect channel 8 (or 16) then
you must send all the preceding ignore's (I). Similarly for the analogue channels where each later entry is
optional if you have already specified the required changes.

e.g.
CC0;NFNFNFNF;I;S1;S2;S3:
CC1;NF:
CC3;NF;S0;S1;S2;S3:

\subsubsection{CD - Configure Digital Input}
\index{Configure Digital Input}
CD\param{chn}; \param{A\textbar{}D\textbar{}S}\param{targetChn}; \param{sp}; \param{actionType}; \param{dwell}; \param{udpType}; \param{AssociatedValue} [; \param{OptValue} [; \param{Options}]]:

Configure one of the digital inputs.
chn is the digital input channel number from 0 to 7 (or more). Remaining parameters are a Trigger Sequence.
Options is a set of flags that controls the digital input event generation.

\index{Options, Digital Input}
\begin{tabular}{l|l|p{12cm}}
Bit&Value&Description\\
\hline
0&1&generate trigger event on rising edge, i.e. 0v to 5v transition.\\
1&2&generate trigger event on falling edge (Default) i.e. 5v to 0v transition, a normally open switch is pressed.\\
2&4&This input and the next is connected to a rotary encoder, when the rotary encoder is turned 'down' this trigger is actioned\\
3&8&\\
4&16&\\
5&32&\\
6&64&\\
7&128&\\
\end{tabular}

To create the option value add the required Value's from the above table together.
For example an option value of 3 (1+2) would set both rising and falling edge flags and this is useful if you have a latching switch of some sort, 
i.e. standard MK light switch.

e.g.
CD7;A1;2;2;0;1;1:
Configure Digital in 7 to target analogue output 1. setpoint 2, action On(2), dwell (0) ignored, UDP

\subsubsection{CE - Configure scheduled event}
\index{Configure scheduled event}
CE\param{chn}; \param{Days}; \param{hours}; \param{Min}; \param{A\textbar{}D\textbar{}S}\param{targetChn}; \param{sp}; \param{actionType}; \param{dwell}; \param{udpType}; \param{AssociatedValue}[; \param{OptValue}];

Configure scheduled event.
chn is event number, currently 16 scheduled events are catered for. Days is a set of day numbers 
from 0-6 to identify which days the event occurs on, e.g. "06" is Sunday and Saturday. hours (0-23) 
and min (0-59) identifies the time within the day when it occurs. 
The remaining parameters constitute a Trigger set as described else where,

e.g.
CE6;12345;12;30;A1;2;2;0;1;1:
Configure event 6, at 12:30 on Mon to Fri to target analogue output 1. setpoint 2, action On(2), dwell (0) ignored, UDP. 

\subsubsection{CI - Configure Analogue Input}
\index{Configure Analogue Input}
CI\param{chn}; \param{L\textbar{}H}; \param{Threshold}; \param{A\textbar{}D\textbar{}S}\param{targetChn}; \param{sp}; \param{actionType}; \param{dwell}; \param{udpType}; \param{AssociatedValue}[; \param{OptValue}];

Configure one of the analogue inputs.
L or H identify whether this is setting a High or Low threshold and Threshold is the analogue threshold
as a value from 0-100\%. The low level hardware is 0-5V but so as to avoid issues when some signal
conditioning is added the input is scaled.

e.g.
CI0;L;20;A1;2;2;0;1;1:

\subsubsection{CM - Configure Mimics}
\index{Configure Mimics}
CM\param{A\textbar{}D}\param{source}; \param{mimicChn}[...]:

Configure mimic channels for one or more analogue and/or digital outputs.

      A target indicates an analogue output number target for a which a mimic channel is specified

      D target indicates an digital output number target for a which a mimic channel is specified

      mimicChn indicates a mimic channel number that will be associated with the corresponding 
      analogue or digital output, to break the connection between a channel and a mimic use a mimic channel
      number of 15.

Pairs of outputs and corresponding mimic channels may be repeated for each association that is 
required to be specified. (Or, a separate command may be used for each.)

The default setting is that Digital outputs 0 to 7 are associated with mimic output 0-7. The off level and
on level can be reconfigured with the SM command.

e.g.
CMA0;0;A1;1;A2;2;A3;3:
target mimics 0 to 3 from analogue 0 to 3.

\subsubsection{CS - Configure Preset point}
\index{Configure Preset point}
CS\param{chn}; \param{val}

Configure one of the set points (Preset point) used for the analogue outputs.
chn is the setpoint number and val is the setpoint value in the range 0-100\%. The low level hardware 
generates 0-10V but this may be passed through some signal conditioning to suit the application.

\subsubsection{CT - Configure Temperature Sensor}
\index{Configure Temperature Sensor}
CT\param{chn}; \param{L\textbar{}H\textbar{}B}; \param{Threshold}; \param{A\textbar{}D}\param{targetChn}; \param{sp}; \param{actionType}; \param{dwell}; \param{udpType}; \param{AssociatedValue}[; \param{OptValue}];

Configure one of the temperature inputs.
L or H or B identify whether this is setting a High or Low or Both threshold(s) and Threshold is the temperature threshold
as a decimal temperature between -50 and +150 degrees Celsius

\subsubsection{CR - Configure Serial}
\index{Configure Serial}
CR\param{mode}; \param{baudIdx}:

The following modes exist:
\begin{tabular}{l|l|p{12cm}}
mode&Description\\
\hline
0&No mode change.\\
2&RS232.\\
4&RS485\\
3&DMX mode\\
\end{tabular}

The following baud rates exist:
\begin{tabular}{l|l|p{12cm}}
index&Speed\\
\hline
0&300\\
1&600\\
2&1200\\
3&2400\\
4&4800\\
5&9600\\
6&19200\\
7&38400\\
8&57600\\
9&115200\\
10&250000\\
11&9600\\
12&9600\\
13&9600\\
14&9600\\
15&9600\\
\end{tabular}

See commands RO, RI, RD

\subsubsection{CW - Configure Dwell}
\index{Configure Dwell}
CW{\it [0-3];DwellValue}:

There are 4 Dwell values that may be set, 0-3.
DwellValue is a number between 2 and 32767, it is measured in 'near seconds'.
Dwell is measured in more or less seconds.
Note that it is the transition between 1 and 0 that marks the end of a Dwell countdown, 
therefore a Dwell value of 10 really gives a Dwell of between 9 and 10 seconds, depending on
when it was started.
\index{Dwell!command sequence}

\subsubsection{DA - Do Attention}
\index{Attention}
DA

Request the WebBrick to send 1 or 2 attention packets, command is generally sent using UDP and broadcast
to all WebBricks on the network.

\subsubsection{DI - Trigger Digital input}
\index{Trigger Digital input}
DI\param{chn}

Trigger input, generates a trigger just as if digital input chn had been triggered.

\subsubsection{DM - Do Mimic}
\index{Do Mimic}
DM\param{mimicChn}; \param{mimicLevel}[; \param{mimicChn}; \param{mimicLevel}]

Set one or more mimic channels to a set level. The mimic channel and level can be repeated to send more than one
request in a single command. This overrides the values set by any recent output change targeted to the mimics
at the WebBricks. mimicLevel can be between 0 and 63.

\subsubsection{DO - Switch Digital output}
\index{Switch Digital output}
DO\param{chn};N\textbar{}F\textbar{}T\textbar{}D[; \param{dwell}]:

Set digital output. Sets the state of one of the digital outputs.
chn is the channel number to operate.

\begin{tabular}{l|p{12cm}}
Action&Description\\
\hline
N&On\\
F&Off\\
T&Toggle\\
D&Dwell\\
\end{tabular}

If Dwell is specified then this can either be a DwellNumber \param{0-7} or if greater than 7 is a dwell in seconds.  If you really want to issue a dwell you're much better
off considering the DT command, this gives you more control from the gateway since you need not worry if the dwell times have been reconfigured at the WebBrick.

\subsubsection{DT - Trigger from External}
\index{Trigger from External}
DT; \param{A\textbar{}D\textbar{}S}\param{targetChn}; \param{sp}; \param{actionType}; \param{dwell}; \param{udpType}; \param{associatedValue}; \param{optValue}[; \param{options}]]:

The packet contains trigger configuration which is actioned on receipt and not stored for later use. This
enables an external source to do what an internally generated event can do. 

{\em NOTES:}

\begin{tabular}{l|p{12cm}}
Parameter&Notes\\
\hline
dwell&This is set in seconds and not an index into the dwell table\\
udpType& this is either 0,None or 1,Send\\
associatedValue&If this is relevant to the action then this can be set\\
optValue&This is not used and should be left at 0\\
options&This is not used and should be left at 0\\
\end{tabular}



\subsubsection{FR - Factory Reset}
\index{Factory Reset}
FR  Configuration Factory reset.

FR1  Full Factory reset.

Perform a factory reset of the WebBrick, the base version only resets user configuration, the later version also resets 
all options and the IP address (Unless IP address locked at Factory).

\subsubsection{IA - InfraRed Address}
\index{InfraRed}
IA\param{address}:

Sets the RC5 infra red address to be recognised. The command values 1-8 are mapped to generating digital triggers, 
i.e. another soft key input.

\subsubsection{InfraRed On Off}
\index{InfraRed}
IR\param{N\textbar{}F}:

Switch on/off infra red reception, uses Digital input 11 and disables any other use for this connection.

\begin{tabular}{l|p{12cm}}
Action&Description\\
\hline
N&On\\
F&Off\\
\end{tabular}

\subsubsection{IS - InfraRed Send}
\index{InfraRed}
IS\param{address}; \param{channel}:

Sends the IR command using RC5 and the address (0-31) and channel (0-63) given. Allows a remote system
to send RC5 infra red commands.


\subsubsection{IT - InfraRed emitter}
\index{InfraRed}
IT\param{N\textbar{}F}:

Switch on/off infra red transmission. Uses DigOut 7 and disables any other use for this connection.

\begin{tabular}{l|p{12cm}}
Action&Description\\
\hline
N&On\\
F&Off\\
\end{tabular}

\subsubsection{LG - Login and Logout}
\index{Login}
LG; \param{password}  

Try to login to the WebBrick to enable the command interface. Up to 3 passwords may be set
Level 1 allows access to the Home page controls, its default is blank so the WebBrick
automatically enters this state.
Level 2 is for reconfiguration
Level 3 is full reconfiguration access for installers.
Note Login times out 5 minutes after the last valid configuration command (Level 3 is 1 Hour timeout). 

Entering an invalid password will logout the user interface.

\subsubsection{NA - Name Analogue Output}
\index{Name Analogue Output}
NA\param{chn}; \param{nameStr}:

Give a name to an analogue output, chn is the channel number from 0 to max analogue outputs-1.

\subsubsection{ND - Name Digital Input}
\index{Name Digital Input}
ND\param{chn}; \param{nameStr}:

Give a name to a digital input, chn is the channel number from 0 to max digital inputs-1.
The name string cannot contain any of the characters '\textless{}\textgreater{}\&

\subsubsection{NI - Name Analogue Input}
\index{Name Analogue Input}
NI\param{chn}; \param{nameStr}:

Give a name to an analogue input, chn is the channel number from 0 to max analogue inputs-1.
The name string cannot contain any of the characters '\textless{}\textgreater{}\&

\subsubsection{NN - Name Node}
\index{Name Node}
\index{NodeName}
NN\param{NodeName}:

Give a name to the WebBrick node.
NodeName is limited to 10 characters  
The name string cannot contain any of the characters '\textless{}\textgreater{}\&

\subsubsection{ND - Name Digital Output}
\index{Name Digital Output}
NO\param{chn}; \param{nameStr}:

Give a name to digital output, chn is the channel number from 0 to max digital outputs-1.
The name string cannot contain any of the characters '\textless{}\textgreater{}\&

\subsubsection{NT - Name Temperature sensor}
\index{Name Temperature sensor}
NT\param{chn}; \param{nameStr}:

Give a name to a temperature input, chn is the channel number from 0 to max temperature inputs-1.
The name string cannot contain any of the characters '\textless{}\textgreater{}\&

\subsubsection{RB - Reboot}
\index{Reboot}
RB  Reboot.

Hardware reboot/reset of the PIC chip and Siteplayer.

\subsubsection{RS - Reboot Siteplayer}
\index{Reboot}
RS  Reboot of Siteplayer.

Hardware reboot/reset of the Siteplayer.

\subsubsection{RT - Re scan 1 Wire}
\index{Re scan 1 Wire}
RT  Re scan 1 wire bus.

Scan the one wire bus for new sensors.

\subsubsection{RU - Refresh User Interface}
\index{Refresh User Interface}
RU  Refresh User Interface.

Resends all data from the PIC chip to the Siteplayer, for use if the PIC chip and Siteplayer are out of step.

\subsubsection{RI - RS485 driver off}
\index{RS485 driver off}
RI:
Set RS485 driver off, input mode

\subsubsection{RO - RS485 driver on}
\index{RS485 driver on}
RO:
Set RS485 driver on, output mode

\subsubsection{RD - Serial send data}
\index{Serial send data}
RD \param{databyte as Decimal ASCII}:
Send serial data, the databyte parameter can be repeated multiple times.

Example, send 'A' character - RD65:  

\subsubsection{SI - Set IP Address}
\index{IP Address}
Set Internet protocol address
SI\param{n}; \param{n}; \param{n}; \param{n}:

Where each \param{n} is an element of the IP address.

SA\param{m}; \param{m}; \param{m}; \param{m}; \param{m}; \param{m}; \param{n}; \param{n}; \param{n}; \param{n}:

Where each \param{m} is an element of the network MAC address.
Where each \param{n} is an element of the IP address.
This is to enable a bunch of WebBricks to be added to a network, identified and addresses set by a discovery process.

\subsubsection{SC - Set Scene}
\index{Set Scene}
SC\param{nr}:

Set the output channels to match a specific scene. The result is the equivalent of issuing On, Off or SetScene 
for any channel not marked as Ignore. NOTE there is a slight difference when a trigger is used to set a scene
in that instead of the ON action for digital channels marked as On and the analogue channels being sent 
the action from the trigger will be sent i.e. Dwell, some possible actions will not make sense.

\subsubsection{SD - Set Date}
\index{Date, Set}
SD\param{years}; \param{mon}; \param{date}  

Set Date for the WebBrick. Not currently implemented or used.
	
\subsubsection{SF - Set FadeRate}
\index{FadeRate, Set}
SF\param{rate}  Set Fade rate

Set the rate at which the analogue channels are adjusted to meet the desired output value. 
The smaller the number the quicker the analogue output channel swings.

With a setting of 1 the analogue will swing full range in approx 200mS, with a value of 255 the swing
full range will take circa 50 seconds.

\subsubsection{SM - Set Mimic}
\index{Mimic, Set}
SM\param{offLevel}; \param{onLevel}; \param{fadeRate}[;\param{0\textbar{}1}]:

Set Mimic high and low level and the fade rate between them. These high and low values are used
when connected as mimics for analogue and digital outputs, by use of an off level that is not quite off 
we have a seek light in the dark. onLevel and offLevel can be between 0 and 63. The fade rate controls the speed that the mimic shifts from one level to
the next, the next level could be selected by the DM command. The final optional parameter sets the mimic 
output to low voltage (approx 4V) or high voltage (approx 10.5V).

\subsubsection{SN - Set NodeNumber}
\index{NodeNumber, Set}
SN\param{NodeNumber}:

Set node number
NodeNumber should be between 1-254, 0 is reserved for 'new' WebBricks that will be configured by
a remote server before they go into production.
\index{NodeNumber}

\subsubsection{SO - Set options flag}
\index{Set options flag}
Set an option flag value, some are bit mapped and others just on/off.
SO\param{num}; \param{value}:

Options control some small bits of a WebBrick's operation.
Most options are not intended for general use and are undocumented.

\begin{tabular}{l|p{12cm}}
Option&Description\\
\hline
1&Manages UDP event transmission.\\
2&Manages Analogue input options.\\
3&Manages Analogue output options.\\
4&Manages Digital input options.\\
5&Manages Digital output options.\\
6&Manages Temperaure sensor options.\\
7&Manages Scene options.\\
\end{tabular}

UDP event transmission option.

\begin{tabular}{l|l|p{12cm}}
Bit&Value&Description\\
\hline
0&1&Enable sending temperature changes.\\
1&2&Enable analogue input changes.\\
2&4&Enable analogue output changes.\\
3&8&Enable Infra red reception UDP packets.\\
4&16&Enable RTC debug UDP packets.\\
5&32&Enable Digital output UDP packets.\\
\end{tabular}

Analogue input options.

\begin{tabular}{l|l|p{12cm}}
Bit&Value&Description\\
\hline
0&1&None.\\
\end{tabular}

Analogue output options.

\begin{tabular}{l|l|p{12cm}}
Bit&Value&Description\\
\hline
1&2&Update DMX channels with analogue output changes.\\
\end{tabular}

Digital input options.

\begin{tabular}{l|l|p{12cm}}
Bit&Value&Description\\
\hline
0&1&None.\\
\end{tabular}

Digital output options.

\begin{tabular}{l|l|p{12cm}}
Bit&Value&Description\\
\hline
0&1&None.\\
\end{tabular}

Temperature sensor options.

\begin{tabular}{l|l|p{12cm}}
Bit&Value&Description\\
\hline
0&1&None.\\
\end{tabular}

Scene options.

\begin{tabular}{l|l|p{12cm}}
Bit&Value&Description\\
\hline
0&1&Scenes setpoints target analogue outputs.\\
1&2&Scenes setpoints target DMX channels.\\
\end{tabular}

Note: scenes can target analogue outputs and/or dmx channels.

\subsubsection{SP - Set password}
\index{Set password}
Set a password.
SP\param{level}; \param{new password}:

level is the password level number 1-3. password is the new password to set. If the same password
is set at multiple levels then login using that password will set itself to the highest of the 
levels using that password string. If Level 1 password is blank the WebBrick will default to being
logged in at level 1 at start and after login timeout. This enables the Home page controls.

\subsubsection{SR - Rotary Encoder Step}
\index{Rotary Encoder Step}
Configure rotary encoder.
SR\param{chn};\param{Steps}:

Steps should be between 2-254, Analogue outputs are set in the raw range 0-1023, where 1023 is 5V, 
Steps controls how far the output is
indexed 'up' or 'down' for each step turn of the rotary encoder. 
The configuration value sets the step for all rotary encoders and chn should be 0. Older webbricks had dedicated pins
for rotary encoders and this only targetted analogue 0, Since 6.4 a rotary encoder may be connected to any even/odd pair of digital 
inputs and target any analogue output or what ever the trigger definition can target.
\index{Rotary Encoder!steps}


\subsubsection{SS - Set Operating Mode}
\index{Operating Mode, Set}
Set operational state
SS\param{ToD}:

ToD is a value between 0-255 that lets the WebBrick know a bit about its operating environment, 
the current recognised values are:

\begin{tabular}{l|p{12cm}}
Value&Description\\
\hline
0&This locks out any commands from the WebBrick holding any outputs at their current state (any Dwell in progress will complete).\\
1&Holiday state, digital inputs ignored.\\
2&This is normal operation.\\
\end{tabular}
\index{ToD}

\subsubsection{ST - Set Time}
\index{Time, Set}
ST\param{dd}; \param{hh}; \param{mm}  Set Time

Set the WebBrick clock. dd is day number from 0-6. hh and mm are the 24 hour time. The WebBrick will
send out starting packets until the clock is set.

\subsubsection{TA - Adjust Analogue Input Threshold}
\index{Adjust Analogue Input Threshold}
TA\param{chn}; \param{L\textbar{}H\textbar{}B}; \param{Threshold};

Configure the active threshold on one of the analogue inputs. This does not update the persistent configuration
only the active configuration.
L or H or B identify whether this is setting a High or Low or Both threshold and Threshold is the analogue threshold
as a value from 0-100\%. The low level hardware is 0-5V but so as to avoid issues when some signal
conditioning is added the input is scaled.

\subsubsection{TT - Adjust Temperature Sensor Threshold}
\index{Adjust Temperature Sensor Threshold}
TT\param{chn}; \param{l\textbar{}H\textbar{}B}; \param{Threshold};

Configure the active threshold on one of the temperature sensors. This does not update the persistent configuration
only the active configuration.
L or H or B identify whether this is setting a High or Low or Both threshold and Threshold is the temperature threshold
as a decimal temperature between -50 and +150 to 1 decimal place degrees Celsius. 
i.e. these are valid 50, 43.3, -43, -54.9. These are not valid 34.95, -43.12.
