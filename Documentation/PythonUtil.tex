\subsection{Utilities}

\begin{description}

\item[WbCfg]

\index{WbCfg configuration download utility}

This program is used to configure a WebBrick from a configuration file.

usage:  \begin{verbatim} python2 WbCfg.py <wbaddress> <wbcfgfile> \end{verbatim}

The configuration file is based on the standard command structure, so to set the node number, the entry would look like:

   \begin{verbatim}N23: \end{verbatim}

You should note that during configuration we set the operating state - to '1' (startup) to disable 
all the functions of the WebBrick.We do this to ensure that there are no extraneous actions or 
commands in progress that would interrupt or corrupt the configuration process.

Here's a partial example of the configuration file:

\verbatiminput{PubPlay.cfg}

\end{description}