\section{CE Marking}

\index{CE Mark}

WebBrick WB10B60 is in conformity with the essential requirements of the directive 1999/5/EC. The product has been tested
 to the standard EN55022:1998 "EMC emmissions and immunity", using the limits described on EN55022 Class B CISPR22(B).


\subsubsection{General Description}


\subsubsection{Risk - High Temperatures}
\index{Risks ! High Temperatures}

The WebBrick has 4 triacs each rated at 2A with an overall rating of 6.3A.  If 6.3A of 240VAC is drawn through the WebBrick
the triacs will dissipate around 7W of heat.  There is sufficient heat sinking in the WebBrick to cope with this.  Internal temperatures
can rise to 85 Deg C.  It is important that the WebBrick is not housed in small totally enclosed areas.  Air circulation is best when the 
WebBrick is mounted in the vertical plane.


\subsubsection{Risk EMI}
\index{Risks ! EMI}

Since the WebBrick uses a number of microprocessors, it may be prone to electromagnetic emissions if it is not installed properly.  The current
standards are stringent and require that a typical installation radiates less that 2 nano watts of power in the RF range.

We recommend that CAT 5/5E/6/7 type cables are used and one core of each pair is connected to ground.

Further we suggest that you use a good quality clean 12V supply to the WebBrick to ensure that supply borne noise is radiated via the ground plane.

It is not a requirement to connect the mains ground to the electronic ground of a WebBrick.  However, please note that if you use the triacs for 
mains loads then the protective mains ground should be connected.  There is no internal connection between this protective ground and the 
electronic ground.


\subsubsection{Risk Software Failures}
\index{Risks ! Software}

The WebBrick contains about 40K assembler code, whilst this has been thoroughly tested, 
software errors cannot be ruled out.

\paragraph{WebBricks should not take the place of safety systems or override safety systems.}

