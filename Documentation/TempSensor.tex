\section {Using the Temperature Sensor}

Here we deal with the basic principles of using the temperature sensors.

\subsection {Hardware}

The temperature sensors are accessed using the DALLAS one wire interface protocol with the master
functionality handled by the PIC chip. The Webbrick6 supports up to 5 DS18B20 temperature sensors. 
These sensors have a 64 bit address that 
can be used to uniquly identify them and they can read to a resolution below 0.1 degrees C.

\subsection {Software}

Every 5 seconds, an attempt is made to read the temperature from all known sensors and update 
the status and values. At start up and on command (RT manual command) the bus will be scanned for new devices. 
When a new sensor is found it is read to see whether it has been tagged previously, if it 
is untagged or the tag corresponds to a currently active sensor then a new tag number is 
generated and stored within the sensor. When a web brick is switched on these tags 
keep the sensors in the same order within the temperature table. The values are all 
available in the status XML.

When a succesful reading is taken it is recorded. Temperature sensor 1 will be tested 
against the Low and High trigger threshold. If the value is outside these values then 
the relevant triggers are processed, these are identical to the digital inout trigger configuration.
 
%\subsection {Practical Application}

%In the real world, where you measure the temperature and where you action the results are 
%often not the same place.

%This is easy to achieve if you have two WebBricks.  You can arrange for one WebBrick to 
%be near where you need to 
%measure temperature and the second, say, in a boiler room.

%The first WebBrick would be configured to trigger a Digital Input channel that was itself configured to generate 
%a remote command.  This remote command would be directed at the second WebBrick.  The second WebBrick would then 
%control the boiler.

%This allows for a network of WebBricks that could all send commands to the 'Boiler WebBrick'.  This would then 
%give a fine degree of control since each WebBrick could have its 'Context' changed to control behaviour, so 
%individual zones could be configured by time to suppress requests to the Boiler WebBrick.  For 'vacation' 
%mode, the Boiler WebBrick could have its 'Context' set to 'Commands Disabled' so that it would turn off 
%the Boiler and not action any requests from the other WebBricks.

\subsection {Installing}
So as to get sensors connected as you want them connect them in the order you want 
them in the temperature table, and wait for each one to register before adding the next sensor.
