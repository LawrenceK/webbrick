\subsection{wb.py the WebBrick class library}

To import this library use the following in your code:

\begin{verbatim} import wb \end{verbatim}

To create a instance of the wb class within you application use:

\begin{verbatim} myName = wb.wb() \end{verbatim}

From this point you can use this instance to interact with WebBricks, for example
if you wanted to check the status of a particular output channel you might use
somethine like

\begin{verbatim} isBoilerOn = myName.DigOutSts(ipAddr,chn) \end{verbatim}

\subsubsection{Functions implemented by wb class}

\begin{description}
	\item[wb] used to instantiate class
		\begin{verbatim} wb.wb() \end{verbatim}
		
	\item[DigOut] used to {\em directly} control a digital output.
		\index{wb Class ! DigOut}
		\begin{verbatim} DigOut(addr,chn,operand) \end{verbatim}
		where:
		\begin{description}
			\item[addr(string)] is the IP address of the WebBrick
			\item[chn(char)] is the chn to be actioned
			\item[operand(char)] is the action, 'F' = OFF, 'N' = ON, 'D'= Dwell\[0\]
		\end{description}

	\item[DigOutSts] used to return the status of a digital output
		\index{wb Class ! DigOutSts}
		\begin{verbatim} result = DigOutSts(addr,chn) \end{verbatim}
		where:
		\begin{description}
			\item[addr(string)] is the IP address of the WebBrick
			\item[chn(char)] is the chn to be actioned
			\item[result(int)] is the 0 for OFF and 1 for ON
		\end{description}
	
	\item[trigger] used to activate a state engine trigger
		\index{wb Class ! trigger}
		\begin{verbatim} result = trigger(addr,trig) \end{verbatim}
		where:
		\begin{description}
			\item[addr(string)] is the IP address of the WebBrick
			\item[trig(char)] is the trigger to be actioned
			\item[result(int)] is 1 for SUCCESS and 0 for FAIL
		\end{description}

	\item[getValue] used to get a value from a WebBrick
		\index{wb Class ! get value}
		\begin{verbatim} result = getValue(addr,name,index) \end{verbatim}
		where:
		\begin{description}
			\item[addr(string)] is the IP address of the WebBrick
			\item[name(string)] is the name of the value to get
			\item[index(int)] is the index of the value, 0 for the first instance of a value
			\item[result(string)] is the value
		\end{description}
			For a full description of value names, look at wbstatus.xml \index{wbstatus.xml}
		
\end{description}
