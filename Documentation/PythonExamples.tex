\subsection{Python Examples}

There are some Python examples that you can use to generate the UDP packets that WebBricks will respond to.  
Please note that in most cases you will need at least version 2.0 of Python for these programs.  This is 
because earlier versions did not like sending broadcast packets on all platforms. If you are an Apple Mac 
user you'll need OSX and Python 2.2 or 2.3, note that 2.3 is bundled with the Panther release of OSX, and 
very nice it is too.


\begin{description}

\item[DigitalOut.py]

This program simply constructs the command string 'DO6T:' which translates to Toggle Digital Out 6.  
This was chosen because on a standard FED PIC development board 4 LEDs are available on PORTD bits 4-7.  
The LED on Bit 7 is currently used as a heart beat with a period of approximately 1 second.  The fact that 
the heartbeat LED flashes tells the user that PIC18F452 program in running and not locked up anywhere.

usage:  \begin{verbatim} python2 DigitalOut.py \end{verbatim}

\item[DigitalCmd]

This program demonstrates the real flexibility of WebBrick server control, since it issues digital output 
commands to any channel.

usage:  \begin{verbatim} python2 DigitalCmd.py -n chn -f chn -t chn -d chn\end{verbatim}

	\begin{description}
	\item[-n chn]  Switch chn oN  
	\item[-f chn]  Switch chn oFf  
	\item[-t chn]  Toggle chn  
	\item[-d chn]  Dwell chn 
	\end{description}




\end{description}
