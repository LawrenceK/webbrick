\usepackage{graphicx, float, fancyheadings, a4wide, makeidx, verbatim,avant,helvet}
\usepackage{}
\pagestyle{fancyplain}
\setlength{\parindent}{0pt}
\setlength{\parskip}{6pt}

%\title{\sf{WebBrick}\linebreak Command syntax}
\lhead{\small{WebBrick Version 7.0 and 6.6}}

\newcommand{\param}[1] {\textless{}#1\textgreater{}}

\rfoot{\tiny{www.WebBrickSystems.com}}
\lfoot{\tiny{\copyright L.P.Klyne}}

\author{Andy Harris, Lawrence Klyne}

\makeindex

\begin{document}

\maketitle

\begin{figure}[H]
\centering
\includegraphics[width=0.3\textwidth]{Images/WebBrickSystems.png}
\end{figure}

\begin{figure}[H]
\centering
\includegraphics[width=0.2\textwidth]{Images/wb_logo.jpg}
\end{figure}

\begin{description}
\item[March 2010 Document Version 7.01]
\item[Firmware Versions 6.6 and 7.0]
\end{description}

\begin{description}
\item[http://www.WebBrickSystems.com] for company information
\item[http://community.WebBrickSystems.com] for WebBrick information
\end{description}

\pagebreak

\tableofcontents

\pagebreak

