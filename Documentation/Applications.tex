\section {Applications}

Here we provide some examples of use of the WebBrick. The great thing with the WebBrick is that a single device can
provide more than one activity. For example PIRs, can trigger video recording at all times. Can switch lights at night. 
Sound warning tones. In typical monitoring systems you need multiple devices because they do not interlink.

\subsection {Heating}

The notes in this section mostly consider that a WebBrick is taking the whole controller role.  Of course it's possible to
add extra global intelligence using a general purpose host computer, for example the WebBrick Gateway.

\subsubsection{Thresholds as set points}
\index{Thresholds}

	In the WeBrick architecture we use temperature sensors rather than thermostats.  We do this for a variety of reasons:
	
	\begin{itemize}
		\item{Controls} - there are no external controls for a temperature sensor.  This means that it can be sited
		where it both makes sense and is convenient.  For example under a kitchen worktop or directly bonded to a heated
		floor.
		\item{Accuracy and repeatability} - Old fashioned bi-metal strip thermostats close at one temperature and open at another.
		Adjusting the set-point is a mechanical operation.  Whereas the sensors we use have resolution to 0.1 deg C and an accuracy
		of 0.5 deg C.
		\item{Remote Control} - Using sensors to read the temperature means that the actions on thresholds are defined within the
		WebBrick.  Therefore the set points can be adjusted on a web page
	\end{itemize}

	With version 6.1 of the WebBrick it is now possible to control thresholds as an action.  Actions can be triggered from Digital Inputs
	and from Schedules.  We can also raise actions from multiple temperature sensors, so we can create zones and frost-stat functionality.
	
	Therefore if we build a heating control scheme using WebBricks we can schedule changes of set-points using thresholds.  We can also
	create {\em Boost} buttons that increase set-points.

\subsubsection{Modern Combi boiler, single Zone}

These generally have power applied all the time which is used to provide hot water and the central heating
is controlled by a mixture of a thermostat and timed events. In this case we would use scheduled events to
control the triacs and relays.  For example the output a triac feeding a relay that is controlled by one
of the temperature sensors using under and over thresholds. In effect the triac replaces the time switch
and the relay replaces the standard thermostat. The great benefits of this are:

	\begin{itemize}
		\item that the control can then be 
		overridden by local buttons for boost (e.g. Dwell 1 hour on boiler output)
		\item the use of a Home gateway system can use its calendar to switch heating off when home is empty. 
		\item a home gateway can access local weather feeds to adjust switch on times dependant on temperature forecasts.
		\item using the home gateway the home owner can control and make changes whilst away from the house, e.g. going home early.
	\end{itemize}

\subsubsection{Two zone heating system}
In this case we drive the boiler off a mains triac that is driven from a schedule entry. And drive two 
zone valves off the two relay outputs, these are triggered by high and low thresholds on two 
temperature sensors. Advantages are as for single Zone.

\subsubsection{Electric under floor heating}
These are great ways of heating floors for things like bathrooms but electricity is not the cheapest
heating source if over used.
So it would be nice for them to come on for timed periods and be able to boost at other times. 
In this case a single output can control the power to
the floor controller, this can be switched by timed events for getting up time and a push button to boost 
it at other times.

Note it is not always best to chain Triac based controllers so if the underfloor heating controller
uses a Triac we recommend using a Relay to power the floor from the WebBrick. Also underfloor heating 
can need power greater than the abilities of the onboard Triacs so an external relay is a good idea.

\subsubsection{Air conditioning}
Controlled by an external relay as the power consumption of these is higher than can be handled directly.
Control by a mix of time schedules and temperature thresholds.

\subsection{Outside lighting/Security Lighting}
This could be as simple as lighting that is timed to come on at dusk until dawn or it could be controlled
by PIR sensors that switch on all the lighting around a house. The PIR sensors may not necessarily be on the same 
WebBrick as the lights. For example on a big building there may be lights on all 4 sides of the building driven by 
different webBricks, on sensing movement on one side you may bring that side up to full brightness and all the other
sides to half brightness.

\subsection{Internal lighting}

\subsubsection{Night Lights}
Especially with the very young and very old it is great to have some low level lighting around a house at 
night, modern LEDs are very good for this drawing very little power. For one or two devices they can be 
switched directly from the digital outputs for heavier loads 3 in series with a resistor can be driven 
from a 12V supply and the open collector driver. If the power supply for a WebBrick is a sealed lead acid (SLA) battery
then this could also provide power for the LED lights so that they stay on during power cuts. With the low power
requirements the system could stay running off a 7 Amp Hour SLA battery for hours if not days.

Maplin (www.maplin.co.uk) 12V 7Ah battery (MG47B) �25, Charger (LL30H) �13.
TLC-direct (www.tlc-direct.co.uk) 12V 6AH battery �10+VAT

\subsubsection{Dimming lights}
For this: use dimmers that accept a 0-10V DC control voltage and wire these to the analogue outputs
from the WebBrick. For example the soundlab G018VA 4 channel dimmer can handle 4 channels of 5A each 
(Max 16A for the Unit).
(See also section on Outside World).

\subsection{Alarm Monitoring}
By connecting dry contacts from fire and burglar alarms to the WebBrick monitor inputs the WebBrick
in association can display the state of the alarm on a central display.The home gateway could then 
send SMS or email messages to alert the home owner. E.g. via www.textanywhere.com.

\subsubsection{Water sensors}
In the case of property in low lying areas water sensors could be installed in basements and under 
floor voids and configured to generate alarms.

\subsection{Household services}

\subsubsection{Water softener}
A sensor could be added to a water softener to detect low salt levels and alert the household through a central
display, especially useful when the device is installed in a basement where it is easily forgotten!

Capacitive sensors are particularly useful for sensing salt levels where the softener has GRP or plastic side walls.

\subsection{Home Cinema}
A home cinema setup can include more than just the audio and video equipment, you may 
also have (for example) roller screens and blackout curtains.

\subsection{Access control}

\subsubsection{Entry gates and doors}
With the use of RF remote systems connected to a WebbRick control of electric gates and garage doors can be handled
through a single system, the WebBrick. You can open gates and doors from the house before braving the elements to get 
access to your car and can then close them from the car as you leave the premises. When video is used as well you can 
open doors from work when your cleaner arrives as you will get a message when the door bell is pushed, you then check the 
camera system verify the identity and let release the front door.

\subsubsection{Camera systems}
A PIR trigger can be set up to start a camera system recording video images which are made available to you on a web interface.
