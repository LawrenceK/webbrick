\section{UDP Output}
\index{UDP Output}

Any trigger (Digital, analogue threshold, temperature threshold, scheduled event, Infra Red receive) 
can be configured to send a 
UDP packet with a configured type indicator, one of General, Remote, Alarm. Output channels also send UDP packets 
to indicate a change has occured this can be used to confirm the operation of remote commands.

The UDP packet format is the same on most occasions (exception being unconfigured node) 
although some fields may not be used and may contain garbage data.

\subsection{General UDP Packet format}

\index{UDP Packet Format}

The UDP Packet Format is formed as follows:

	\begin{tabular}{l|p{12cm}}
	Field&Description\\
    \hline
	Len (byte)&The overall size of the UDP packet.\\
	udpType (byte)&  The packet type one of the characters 'A', 'G', 'R'.\\
	Source 0-1 (2bytes)& These are two characters to identify the Udp source see later table.\\
	srcChannel (byte)& The channel/event index.\\
	tgtChannel (byte)& The target channel when sending remote commands, if the target channel type is analogue
			then the top bit is set, i.e. values are 128+channel.\\
	action (byte)& The action being triggered (low 4 bits), includes Dwell number (high 4 bits).\\
	fromNodeNr (byte)& Source webrick number 1-253.\\
	toNodeNr (byte)& Target webrick number for remote commands.\\
	setPointNr (byte)& Where relevant for packet type.\\
	curValH (byte)& High byte of any value being sent\\
	curValL (byte)& Low byte of any value being sent\\
	\end{tabular}

\subsubsection{UDP Packet Types}
	
The UDP Packet Types are:

\index{UDP Packet Types}
	\begin{tabular}{l|p{12cm}}
	Type&Description\\
    \hline
	G&General format UDP Packet generated\\
	\end{tabular}

    There used to be a Remote and Alarm type in pre 6.6 firmware, this was never used in any applications and the space used to define these has been recovered for other uses.
    
\subsubsection{UDP Packet Sources}
	
The UDP Packet Sources are:

\index{UDP Packet Source}
	\begin{tabular}{l|p{12cm}}
	Code&Description\\
    \hline
	Ta&Low analogue threshold trigger\\
	TA&High analogue threshold trigger\\
	Td&Trigger from remote DI command\\
	TD&Trigger from local digital input\\
	Tt&Low temperature threshold trigger\\
	TT&High temperature threshold trigger\\
	TS&Trigger from scheduled event\\
	TR&Trigger from infra red remote control\\
	TX&Trigger from external source, i.e. WebBrick Gateway\\
	IR&infra red remote control\\
	AI&New analogue input value 0-100\\
	AO&New analogue output value 0-100\\
	CT&Current temperature in 1/16ths degree\\
	DO&Digital output/monitor state.\\
	NN&Unconfigured node.\\
	SS&Node starting, clock not set.\\
	AT&Atention button pressed on webbrick, AKA Factory Reset button.\\
	AA&Dicovery triggered response, Alert.\\
	CC&Configuration changed.\\
	\end{tabular}

Further details on each packet follow.
	
Trigger
\index{Low analogue threshold trigger}
\index{High analogue threshold trigger}
\index{Low temperature threshold trigger}
\index{High temperature threshold trigger}
\index{Trigger from scheduled event}
\index{Trigger from local digital input}
\index{Trigger from remote DI command}
\index{Trigger from external source}
\index{Monitor state}

For this packet format:

\begin{enumerate}
\item the srcChannel identifies the channel number or scheduled event number. In the case of the monitor
inputs this is the monitor number + number of digital inputs. In a later revision of the firmware the monitor inputs will
just be other digital inputs and can be a normal trigger. In the case of external source this will be zero.
\item  the tgtChannel is the remote webrick node number from the trigger configuration
\item  udpType is from the trigger configuration
\item  action is from the trigger configuration action and dwell number if the trigger type is analogue
\item  fromNodeNr is my node number
\item  toNodeNr is from the trigger configuration and is only relevant when the udpType is 'R' remote.
\item  SetPointNr comes from trigger configuration.
\item  CurValH is not used.
\item  CurValL is used for local digital inputs and is an estimate of the number of times the input is triggered
	in the last second.\\
\end{enumerate}

\subsubsection{UDP - Analogue Values}
Analogue values

\index{New analogue input value}
\index{New analogue output value}

For this packet format:
\begin{enumerate}
\item  the srcChannel identifies the analogue channel number
\item  the tgtChannel is not used
\item  udpType is 'G' General
\item  action is not used
\item  fromNodeNr is my node number
\item  toNodeNr is not used
\item  SetPointNr is not used
\item  CurValH is not used.
\item  CurValL is the new value for the input or output in the range 0-100.\\
\end{enumerate}


\subsubsection{UDP - Digital Output and Input States}
	
Digital/Monitor State

\index{Digital output}

For this packet format:
\begin{enumerate}
\item  the srcChannel identifies the digital output channel number.
\item  the tgtChannel is not used
\item  udpType is 'G' General
\item  action is either DINACTIONON or DINACTIONOFF
\item  fromNodeNr is my node number
\item  toNodeNr is not used
\item  SetPointNr is not used
\item  CurValH is not used.
\item  CurValL is not used.\\
\end{enumerate}

\subsubsection{WebBrick Starting}

WebBrick node starting

\index{Node starting}
\index{Clock not set}

For this packet format:
\begin{enumerate}
\item  fromNodeNr is my node number
\end{enumerate}
All other fields are not used.\\
	
\subsubsection{UDP - Unconfigured Node}

Unconfigured

\index{Unconfigured node}

For this packet format the payload does not follow the standard format, the first 6 bytes of the payload are 
the MAC address of the network interface.\\

\subsection{WebBrick Time and UpTime format}

\index{WebBrick Uptime}

This packet has the following format:

\begin{enumerate}
\item  the srcChannel identifies the digital output channel number.
\item i the tgtChannel is not used
\item  udpType is 'G' General
\item  'S'
\item  'T'
\item  WebBrick Time - HOURS, taken from the Real-Time Clock
\item  WebBrick Time - MINUTES, taken from the Real-Time Clock
\item  WebBrick Time - SECONDS, taken from the Real-Time Clock
\item  WebBrick Time - DAY, taken from the Real-Time Clock
\item  WebBrick Uptime, upper byte
\item  WebBrick Uptime, lower byte
\item  ResetCode\\
\end{enumerate}

ResetCode can have the following values:

\begin{enumerate}
\item  '0' This means that the SitePlayer (WebServer) was reset by either the internal watchdog, or a manual 'RS' command
\item 'value of RCON', this is the internal value of the Reset Register of the WebBrick processor (Normally '92' for 6 and 7 series WebBricks) \\
\end{enumerate}




