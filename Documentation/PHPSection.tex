\section{WebBrick and PHP}

\subsection{Introduction}

PHP is a web application langauge that allows you to mix business logic and user interface logic in a single file, this file
is processed on a web server to mix fixed and variable parts of the user interface. http://www.php.net

For the WebBrick we provide a number of PHP resources, these fall into the following catergories:

\begin{tabular}{l|p{12cm}}
    \hline
	Libraries&These allow the user to create programs that interact with WebBricks
				using routines that have been fully tested and debugged.  This allows
				the programmer to get on with building the user interface they desire rather
				than getting bogged down in the details of network protocols.\\
    \hline
	Example Code&This allows people to quickly build on some of the ideas we already had and
				implemented.\\
\end{tabular}

THESE PHP resources are not actively maintained and are provided AS IS with no support, they are similar to the python
code so refer to the python section for details. Your PHP will need to have an XML parser as part of it's install for the
status and configuration classes to work.

\subsection{Libraries}

\subsubsection{wb.php basic comms libray}
\index{wb.php}

This php module provides basic communication with the WebBrick and is mainly intended to be a building block.

To use this library use the following in your code:

\begin{verbatim} include("../API/wb.php") ;\end{verbatim}

From this point you can use this module to interact with WebBricks, for example
if you wanted to switch on a particular output channel you might use
somethine like

wbSendCmd( '10.100.100.100', 'DO3;N' )

\subsubsection{wb6.php the WebBrick 6 library}
\index{wb6.php}

This module provides basic commands that can be sent to a WebBrick.

To import this library use the following in your code:

\begin{verbatim} include("../API/wb6.php") ;\end{verbatim}

From this point you can use this module to interact with WebBricks, for example
if you wanted to switch on a particular output channel you might use
somethine like

wb6DigOn( '10.100.100.100', 3 )


\subsubsection{wb6Status.php}
\index{wb6Status.php}

This class retrieves a copy of the current WebBrick Status that can then be asked for specific details. Note
the object reads and caches the values, so you get a current snapshot. To get updated values use the load method.
There is no point retrieveing tha values at more a 1 second interval and it is recommended that you do as much as possible
with a single retrieved snapshot.


\subsubsection{wb6Cfg.php}
\index{wb6Cfg.php}
This class retrieves a copy of the current WebBrick configuration that can then be asked for specific details.

Trigger results are returned as an asociative array with some of the following parameters in it, refer to commands 
section for details on these values.

\begin{tabular}{l|p{12cm}}
    attribute name&Description\\
    \hline
    Name&channel name\\
    Value&threshold value\\
    Hours&scheduled time\\
    Minutes&scheduled time\\
    Days&scheduled days\\
    actionNr&action type number\\
    action&action type string\\
    dwell&dwell number\\
    UDPRemNr&UDP packet type number\\
    UDPRem&UDP packet type string\\
    RemNode&target remote node.\\
    typeNr&channel type number\\
    type&channel type string\\
    setPoint&setpoint number\\
    pairChn&target channel number\\
\end{tabular}

Scene results are returned as an asociative array with the following parameters in it.

\begin{tabular}{l|p{12cm}}
    attribute name&Description\\
    \hline
    Digital\param{nn}&Ignore|On|Off\\
    Analogue\param{nn}&Ignore|SetPoint\param{mm}\\
\end{tabular}
where \param{nn} is a channel number and \param{mm} is a setpoint number.

\subsection{PHP Examples}

There is a PHP example that you can modify to produce your own central user interface.  

\subsubsection{PanelLib.php}
\index{PanelLib.php}

This provides a library of code for a user interface panel that is capable of displaying details from one or more 
webBricks and issuing commands to them. This uses an Xml file standard.xml to define a user interface that consists 
of a number of columns that contain status information and/or commands.

The xml file has a top level element called status with a single attribute cols that specifies the number of columns on the
display. There are then a set of elements called item. Each item has a Type element and typically a channel number and an
IP address to identify the webbrick and the data to access from it. Each entry may also contain a Trig element that is used to
specify a command to be issued.

item Type may be one of:

\begin{tabular}{l|p{12cm}}
    type&Description\\
    \hline
    Header&generates a text of Cols width.\\
    Temp&displays a temperature.\\
    inAnalogue&displays an analogue input value.\\
    Analogue&displays an analogue output.\\
    Switch&displays a digital output state.\\
    Monitor&displays a monitor state.\\
\end{tabular}

\subsubsection{standard.php}
\index{standard.php}

This is the start page for the standard panel.

